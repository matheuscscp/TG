Boolean satisfiability is a problem of great practical interest and the first problem ever to be stated as NP-complete. Many algorithms for this problem are based on normal forms. Also, naive translations to normal forms may lead to exponential growth on the size of a formula, a highly undesirable effect. It is known, however, that techniques such as formula renaming may help reducing the size of such translation. In this work, we present a heuristic to find renamings that may reduce the size of translated formul\ae \; and a dynamic programming algorithm to compute the renaming given by the heuristic, together with its proof of correctness, complexity analysis, and experimental results. We also present comparison results with a greedy algorithm which is optimal for a restricted class of the problem. Empirical evaluation shows that, in most cases, the proposed heuristic produces as fewer clauses as the greed algorithm for this restricted class of the problem and behaves even better for some specific unrestricted inputs.
