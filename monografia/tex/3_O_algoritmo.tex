
\indent

Neste capítulo, apresentamos um algoritmo que encontra um renomeamento ótimo para qualquer fórmula.

\begin{lemma}
	Sejam $\phi$ e $\psi$ fórmulas tais que $\psi \sqsubset \phi$. Então existem inteiros $m$ e $n$ não negativos e não simultaneamente nulos tais que $$p(\phi) = p_0 + m p(\psi) + n \overline{p}(\psi)$$ onde $p_0$ é .
\end{lemma}

\emph{Prova}: Primeiro, observe que qualquer fórmula gera no mínimo uma cláusula, ou seja, $p(\psi) > 0, \forall \psi \in \mathcal{L}$. Da Tabela \ref{numerop}, observe também que as únicas operações aritméticas que podem aparecer em uma expressão para $p(\phi)$ são a adição e a multiplicação. Isto garante que não podem aparecer operandos negativos em qualquer expressão para $p(\phi)$.

Considere uma expressão para $p(\phi)$ tal que $p(\xi)$ só não ocorre se $s$

Considere então uma expressão tal que Aplicando então distribuição, agrupando os termos em que $p(\psi)$ é um fator e colocando $p(\psi)$ em evidência, temos $$p(\phi) = p_0' + m p(\psi)$$ Fazendo o mesmo para os termos em que $\overline{p}(\psi)$ é um fator, obtemos $$p(\phi) = p_0 + m p(\psi) + n p(\psi)$$

Note que não pode ocorrer $m = n = 0$, pois, neste caso, $\psi$ não seria subfórmula de $\phi$.

\begin{theorem}
	\label{optimal_sub}
	Seja $SFP(\phi) = \{\phi_1,...,\phi_n \}$, onde $\phi_i \sqsubset \phi_j \implies i < j$, e seja\break $R = \{\phi_{k_1},...,\phi_{k_r} \} \subseteq SFP(\phi)$ um renomeamento ótimo que considere somente entre os que contêm no\break máximo $j$ subfórmulas e consideram somente
	Seja $R \subseteq SFP(\phi)$ um renomeamento ótimo entre os que contêm no máximo $j$ subfórmulas e seja $\psi \in R$. Então, permitindo que só as subfórmulas em $SFP(\phi) - \{\psi \}$ sejam escolhidas, $R - \{\psi\}$ é um renomeamento ótimo entre os que contêm no máximo $j-1$ subfórmulas.
\end{theorem}

\emph{Demonstração}: Seja $R_1 = R - \{\psi\}$ e, para obter uma contradição, suponha que $R_1' \subseteq SFP(\phi) - \{\psi \}$ é um renomeamento com no máximo $j-1$ subfórmulas ainda melhor que $R_1$, ou seja, denotando por $p(R)$ o número de cláusulas em uma transformação de renomeamento que usa $R$, temos que $R_1'$ é tal que $p(R_1') < p(R_1)$. Então, assim como $R$, o renomeamento $R' = R_1' \cup \{\psi \}$ contém no máximo $j$ subfórmulas.

Para cada um dos quatro renomeamentos definidos, vamos agora expressar o número de cláusulas que cada um gera, usando um resultado de Boy de la Tour \cite{de1992optimality}. Temos que
\begin{enumerate}
	\item $p(R) = p_0 + a + p(\psi) + b + \overline{p}(\psi)$
	\item $p(R_1) = p_0 + a p(\psi) + b \overline{p}(\psi)$
	\item $p(R_1') = p_0' + a' p(\psi) + b' \overline{p}(\psi)$
	\item $p(R') = p_0' + a' + p(\psi) + b' + \overline{p}(\psi)$
\end{enumerate}




Portanto, se concluirmos que $p(R') < p(R)$, então teremos uma contradição, pois sabemos que $R$ é ótimo por hipótese.

\section{Uma fórmula recursiva}

\indent

Seja $SFP(\phi) = \{\phi_1,...,\phi_n \}$ e defina $f(i,j)$ como um renomeamento ótimo que contém no máximo $j$ subfórmulas, considerando somente as subfórmulas em $\{\phi_1,...,\phi_i \}$.

Por definição, é claro que $f(i,0) = f(0,j) = \emptyset, \forall i,j$.

Agora, note que, ou $\phi_i \in f(i,j)$, ou $\phi_i \notin f(i,j)$.

Suponha que $\phi_i \in f(i,j)$. Neste caso, o Teorema \ref{optimal_sub} nos garante que $f(i,j) - \{\phi_i \}$ é um renomeamento ótimo com no máximo $j-1$ subfórmulas, considerando somente as subfórmulas em $\{\phi_1,...,\phi_{i-1} \}$, ou seja, $p(f(i,j) - \{\phi_i \}) = p(f(i-1,j-1))$.

Suponha agora que $\phi_i \notin f(i,j)$. Neste caso, é claro que $p(f(i,j)) = p(f(i-1,j))$.

Portanto, podemos definir $f(i,j)$ da seguinte maneira. Se $i > 0$ e $j > 0$, então
\[
f(i,j) =
\begin{cases} 
\hfill f(i-1,j-1) \cup \{\phi_i \}   \hfill & \text{ se } p(f(i-1,j-1) \cup \{\phi_i \}) < p(f(i-1,j)) \\
\hfill f(i-1,j) \hfill & \text{ caso contrário} \\
\end{cases}
\]
onde é fácil ver que não há dependência cíclica.

Para obter então um renomeamento ótimo considerando todas as subfórmulas próprias de $\phi$, basta calcular $f(n,n)$.

A próxima seção apresenta um algoritmo baseado em programação dinâmica \cite{bellman2015applied} para calcular $f(n,n)$.

\section{Uma implementação por computação ascendente}

\indent

O Algoritmo \ref{knapsack} calcula $f(n,n)$ por computação ascendente, ou seja, calcula $f(i,j)$, para todo $j$, primeiro para valores pequenos de $i$, até finalmente calcular $f(n,n)$.

\begin{algorithm}
	\begin{algorithmic}[1]
		\State seja $dp[0..n]$ um novo arranjo com $dp[j] = \emptyset$ para todo $j$
		\For{$i$}{$1$}{$n$}
			\For{$j$}{$n$ \textbf{descendo}}{$1$}
				\State $alt \gets dp[j-1] \cup \{\phi_i\}$
				\If{$p(alt) < p(dp[j])$}
					\State $dp[j] \gets alt$
				\EndIf
			\EndFor
		\EndFor
	\end{algorithmic}
	\caption{Computação ascendente de $f(n,n)$.}
	\label{knapsack}
\end{algorithm}

\subsection{Prova de correção}

\indent

A correção do Algoritmo \ref{knapsack} segue das invariantes de laço a seguir.

\textbf{Invariante 1}: No início de cada iteração do laço \textbf{para} das linhas 2--9, temos que $dp[j] = f(i-1,j), \forall j \leq n$.

\textbf{Inicialização da Invariante 1}: No início da primeira iteração do laço, temos que $i = 1$ e $dp[j] = \emptyset = f(0,j) = f(i-1,j), \forall j \leq n$, logo a invariante vale.

\textbf{Manutenção da Invariante 1}: Suponha que, antes de uma iteração do laço,\break $dp[j] = f(i-1,j)$, $\forall j \leq n$. Para provar que o mesmo vale antes da iteração seguinte, enunciamos uma segunda invariante.

\textbf{Invariante 2}: No início de cada iteração do laço \textbf{para} das linhas 3--8:
\begin{enumerate}
	\item Se $k \leq j$, então $dp[k] = f(i-1,k)$.
	\item Se $j < k \leq n$, então $dp[k] = f(i,k)$.
\end{enumerate}

\textbf{Inicialização da Invariante 2}: No início da primeira iteração do laço, temos que $j = n$. Pela hipótese de manutenção da Invariante 1, antes da primeira iteração do laço, temos que $dp[k] = f(i-1,k)$, $\forall k \leq n = j$. Portanto, o item 1 da invariante vale. Além disso, o item 2 é satisfeito por vacuidade, pois, $\forall k > j$, temos que $k > n$.

\textbf{Manutenção da Invariante 2}: Suponha que, antes de uma iteração do laço, os itens 1 e 2 da invariante sejam verdade. Neste caso, após a linha 4, temos que\break $alt = dp[j-1] \cup \{\phi_i \} = f(i-1,j-1) \cup \{\phi_i \}$. Além disso, $dp[j] = f(i-1,j)$.\break Portanto, a linha 6 só será executada se $p(f(i-1,j-1) \cup \{\phi_i \}) < p(f(i-1,j))$. Se isto ocorre, então $f(i,j) = f(i-1,j-1) \cup \{\phi_i \}$ e, após a linha 7, $dp[j] = f(i,j)$. Se isto não ocorre, então $f(i,j) = f(i-1,j)$ e, após a linha 7, $dp[j] = f(i,j)$. Portanto, sob qualquer hipótese, temos que $dp[j] = f(i,j)$ após a iteração do laço, o que prova que os itens 1 e 2 da invariante irão valer na iteração seguinte.

\textbf{Terminação da Invariante 2}: A condição para que o laço termine é $j < 1$. Como cada iteração subtrai 1 de $j$, em algum momento teremos $j = 0$, ou seja, o laço termina. Além disso, mostramos que, neste ponto, $dp[k] = f(i,k), \forall k > j = 0$, ou seja,\break $dp[j] = f(i,j), \forall j$.

A propriedade provada na terminação da Invariante 2 prova que, se a Invariante 1 vale antes de uma iteração, então ela também irá valer na iteração seguinte, o que conclui a manutenção da Invariante 1.

\textbf{Terminação da Invariante 1}: A condição para que o laço termine é $i > n$. Como cada iteração adiciona 1 a $i$, em algum momento teremos $i = n+1$, ou seja, o laço termina. Além disso, mostramos que, neste ponto, $dp[j] = f(i-1,j) = f(n,j), \forall j$, ou seja,\break $dp[n] = f(n,n)$.

Note que o algoritmo computa não somente o renomeamento ótimo $f(n,n)$, mas também os renomeamentos ótimos para versões mais restritas do problema. Por exemplo, se quisermos permitir que no máximo $j < n$ subfórmulas sejam escolhidas, basta utilizar o resultado $dp[j] = f(n,j)$.

\subsection{Análise}

\indent

O custo de tempo do Algoritmo \ref{knapsack} é o custo das linhas 4--7 multiplicado por $n^2$. A linha 4, que cria um conjunto de no máximo $n$ elementos e acrescenta a ele um novo elemento, custa $O(n)$ no pior caso. A linha 5 custa o tempo de calcular o número de clásulas de duas transformações de renomeamento. É possível mostrar que este custo é $O(|pos(\phi)|)$ no pior caso \cite{nonnengart2001computing}. A linha 6 custa o tempo de copiar e destruir um conjunto com no máximo $n$ elementos, ou seja, $O(n)$ no pior caso. Por fim, temos que $n \leq |pos(\phi)|$. Portanto, o custo das linhas 4--7 é $O(2n + 2|pos(\phi)|) = O(|pos(\phi)|)$ no pior caso; e o custo de tempo total do algoritmo é $O(n^2 |pos(\phi)|) = O(|pos(\phi)|^3)$ no pior caso.

O custo de espaço do Algoritmo \ref{knapsack} é dado pela soma dos custos do arranjo $dp$, da variável $alt$ e do custo de espaço da função $p(R)$. O arranjo $dp$ contém $n+1$ conjuntos de no máximo $n$ elementos, logo seu custo de espaço é $O(n^2)$ no pior caso. A variável $alt$ é um conjunto de no máximo $n$ elementos, logo seu custo é $O(n)$ no pior caso. É possível mostrar que o custo de espaço para calcular $p(R)$ é $O(|pos(\phi)|)$ no pior caso \cite{nonnengart2001computing}. Portanto, o custo de espaço total do algoritmo é $O(n^2 + n + |pos(\phi)|) = O(|pos(\phi)|^2)$ no pior caso.
