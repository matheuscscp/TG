\indent

Lógicas têm sido utilizadas em Computação para representar e raciocinar sobre problemas. A representação se dá através de linguagens formais. formulas... bem formadas... sintaxe e semantica.

Satisfatibilidade, o problema de determinar se existe uma interpretação sob a qual uma afirmação é verdadeira, é de grande interesse prático. Tal situação aparece, por exemplo, em vários problemas da microeletrônica, como síntese \cite{bloem2014sat}, otimização \cite{gupta2006sat} e verificação \cite{nieuwenhuis2006sat} de \textit{hardware}. Aparece também em problemas de raciocínio automático \cite{harrison2009handbook} e em muitos outros problemas relevantes \cite{horvitz1992automated}.

Satisfatibilidade é também de grande interesse teórico. Em 1971, Cook definiu a classe dos problemas NP-completos, sendo satisfatibilidade proposicional um dos problemas representativos desta classe. Ele também formalizou o enunciado do maior problema ainda não resolvido da Ciência da Computação: P versus NP \cite{cook1971complexity}.

Grande avanço já foi feito em direção a algoritmos de busca rápidos para satisfatibilidade \cite{davis1960computing,davis1962machine,biere2009conflict}, apesar de ser conjecturado que qualquer um terá custo de tempo exponencial determinístico no pior caso.

Uma característica comum a diversos dos algoritmos para satisfatibilidade já descobertos é a redução do problema a fórmulas em uma determinada forma normal. Resultados obtidos com tal e tal algoritmo mostram que é possível lidar melhor com problemas de satisfatibilidade, se esta redução for utilizada. Nestes casos, etapas de pré-processamento são necessárias para colocar fórmulas quaisquer nas respectivas formas normais.

Considerando a possibilidade de melhorar a eficiência total através de pré-processa\-mento, pesquisas já foram conduzidas para verificar a possibilidade de poupar esforço computacional durante a execução do algoritmo de busca, ao executar um pré-processamento para encontrar fórmulas menores. Em particular, tal e tal pessoa tentam responder a esta pergunta, utilizando fórmulas na forma normal clausal. Tal pessoa emprega a heurística de que fórmulas com menos cláusulas produzem respostas mais rapidamente.

Ao procurar algoritmos de pré-processamento, é importante buscar aqueles com custos significativamente menores que os dos algoritmos de busca. Isto é claro, pois não há vantagem em executar um pré-processamento de custo igualmente exponencial. Por exemplo, procuramos pré-processamentos de custos polinomiais.

Este trabalho compara técnicas baseadas em renomeamento, aplicadas para minimizar a quantidade de cláusulas na conversão de uma fórmula para a forma normal clausal. É proposto um algoritmo polinomial de programação dinâmica para determinar o renomeamento que gera menos cláusulas, limitando o número de subfórmulas renomeadas.

Aqui na introdução, falar de resultados e trabalhos futuros?
