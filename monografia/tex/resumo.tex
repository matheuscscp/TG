Satisfatibilidade booleana é um problema de grande interesse prático, além de ser o primeiro problema provado ser NP-completo. Desde que ganhou atenção, diversos algoritmos para este problema já foram descobertos e alguns deles são baseados em formas normais. As transformações usuais para formas normais podem acarretar aumento exponencial do tamanho da fórmula, o que não é desejável. No entanto, é sabido que técnicas como renomeamento podem auxiliar a reduzir o tamanho da fórmula transformada. Neste trabalho, apresentamos uma heurística para o problema de encontrar renomeamentos que reduzem o tamanho de fórmulas e um algoritmo baseado em programação dinâmica para calcular o renomeamento dado pela heurística, junto com sua prova de correção, análise de complexidade e resultados experimentais. Ainda comparamos o algoritmo proposto com um algoritmo guloso para o mesmo problema.