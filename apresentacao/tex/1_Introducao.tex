
\section{Introdução}

\begin{frame}{Lógica}
	\begin{itemize}
		\item Lógicas são utilizadas para representar e raciocinar sobre problemas computacionais.
		\pause\item A representação se dá através de uma linguagem formal, de \emph{fórmulas}.
		\pause\item Para atribuir um significado a cada fórmula, define-se para a lógica uma \emph{semântica}\pause, que possui diferentes \emph{interpretações}.
		\pause\item Em lógicas clássicas, os significados possíveis são somente \emph{verdadeiro} ou \emph{falso}.
	\end{itemize}
\end{frame}

\begin{frame}{SAT}
	\emph{Satisfatibilidade}: Determinar se existe uma interpretação sob a qual uma dada fórmula é verdadeira.
	\begin{itemize}
		\pause\item Possui grande interesse prático:
		\begin{itemize}
			\pause\item Síntese \cite{bloem2014sat}, otimização \cite{nieuwenhuis2006sat} e verificação \cite{gupta2006sat} de \textit{hardware}.
			\pause\item Raciocínio automático \cite{harrison2009handbook}.
			\pause\item Biologia e medicina \cite{horvitz1992automated}.
		\end{itemize}
		\pause\item Interesse teórico fundamental:
		\begin{itemize}
			\pause\item Primeiro problema NP-completo \cite{cook1971complexity}.
			\pause\item Deu base para formalizar P \textit{versus} NP \cite{cook1971complexity}.
		\end{itemize}
	\end{itemize}
\end{frame}

\begin{frame}{VAL}
	\emph{Validade}: Determinar se uma dada fórmula é verdadeira sob qualquer interpretação.
	
	\vspace{.5cm}
	\pause SAT e VAL são redutíveis um ao outro!
\end{frame}

\begin{frame}{Algoritmos para SAT e VAL}
	\begin{itemize}
		\item Há diversos algoritmos de busca para SAT e VAL \cite{davis1960computing,davis1962machine,biere2009conflict}.
		\pause\item Conjectura-se que todos são exponenciais \cite{cook1971complexity}.
		\pause\item Muitos são baseados em \emph{formas normais}: subconjuntos de fórmulas.
		\pause\item Algoritmos baseados em formas normais precisam de pré-processamento eficiente.
	\end{itemize}
\end{frame}

\begin{frame}{O trabalho}
	\begin{block}{Hipótese}
		Considerando melhorar a eficiência total de pré-processamento e busca: fórmulas menores produzem respostas mais rápido?
	\end{block}
	\pause
	\begin{block}{Objetivo}
		Testar a hipótese experimentalmente.
	\end{block}
\end{frame}

\begin{frame}{O trabalho}
	\begin{itemize}
		\item Investigamos algoritmos baseados na \emph{forma normal clausal}.
		\pause\item Tentamos obter fórmulas pequenas reduzindo o \emph{número de cláusulas}\pause, através de \emph{renomeamento}.
		\pause\item Boy de la Tour \cite{de1992optimality} e Jackson et al. \cite{jackson2004clause} propõem algoritmos para este problema.
		\pause\item Propomos um algoritmo baseado em programação dinâmica para este problema.
		\pause\item Comparamos experimentalmente o algoritmo que propomos com o de Boy de la Tour.
	\end{itemize}
\end{frame}
