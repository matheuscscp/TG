
\section{Introdução}

\begin{frame}{Motivação}
	\begin{itemize}
		\item SAT é de grande interesse prático \pause e de interesse teórico fundamental.
		\pause\item Conjectura-se que não exista solução polinomial.
		\pause\item A maioria dos algoritmos se baseia em formas normais.
		\pause\item Tais algoritmos precisam de pré-processamento\pause, que precisa:
		\begin{itemize}
			\pause\item ser rápido\pause; e
			\item produzir fórmulas ``pequenas''.
		\end{itemize}
	\end{itemize}
\end{frame}

\begin{frame}{O trabalho}
	\begin{block}{Hipótese}
		Considerando melhorar a eficiência total de pré-processamento e busca: fórmulas menores produzem respostas mais rápido?
	\end{block}
	\pause
	\begin{block}{Objetivo}
		Testar a hipótese experimentalmente.
	\end{block}
\end{frame}

\begin{frame}{O trabalho}
	\begin{itemize}
		\item Investigamos algoritmos baseados na \emph{forma normal clausal}.
		\pause\item Tentamos obter fórmulas pequenas reduzindo o \emph{número de cláusulas}\pause, através de \emph{renomeamento} \cite{plaisted1986structure}.
		\pause\item Boy de la Tour \cite{de1992optimality} e Jackson et al. \cite{jackson2004clause} propõem algoritmos para este problema.
		\pause\item Propomos um algoritmo baseado em programação dinâmica para este problema.
		\pause\item Comparamos experimentalmente o algoritmo que propomos com o de Boy de la Tour.
	\end{itemize}
\end{frame}
