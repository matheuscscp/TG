
\documentclass{beamer}

%%%%%%%%%%%%%%%%%%%%%%%%%%%%%%%%%%%%%%%%%%%%%%%%%%%%%%%%%%%%%%%%%%%%%%
% Codificação
%%%%%%%%%%%%%%%%%%%%%%%%%%%%%%%%%%%%%%%%%%%%%%%%%%%%%%%%%%%%%%%%%%%%%%

\usepackage[brazilian]{babel}
\usepackage[utf8]{inputenc}
\usepackage{lmodern}
\usepackage{mathbbol}
\usepackage{multicol}
\usepackage{algorithmic}
\usepackage{tikz}
\usepackage{pgfplots}


% Declaracoes em Português
\renewcommand\algorithmicend{\textbf{fim}}
\renewcommand\algorithmicdo{\textbf{faça}}
\renewcommand\algorithmicwhile{\textbf{enquanto}}
\renewcommand\algorithmicfor{\textbf{para}}
\renewcommand\algorithmicif{\textbf{se}}
\renewcommand\algorithmicthen{\textbf{então}}
\renewcommand\algorithmicelse{\textbf{senão}}
\renewcommand\algorithmicreturn{\textbf{retorne}}
\renewcommand\algorithmicrequire{\textbf{Entrada:}}
\renewcommand\algorithmicensure{\textbf{Variáveis:}}

%%%%%%%%%%%%%%%%%%%%%%%%%%%%%%%%%%%%%%%%%%%%%%%%%%%%%%%%%%%%%%%%%%%%%%
% Tema do beamer
%%%%%%%%%%%%%%%%%%%%%%%%%%%%%%%%%%%%%%%%%%%%%%%%%%%%%%%%%%%%%%%%%%%%%%

\usetheme{Warsaw}

\defbeamertemplate*{footline}{shadow theme}{%
\leavevmode%
\hbox{\begin{beamercolorbox}[wd=.5\paperwidth,ht=2.5ex,dp=1.125ex,leftskip=.3cm plus1fil,rightskip=.3cm]{author in head/foot}%
    \usebeamerfont{author in head/foot}\hfill\insertshortauthor
\end{beamercolorbox}%
\begin{beamercolorbox}[wd=.4\paperwidth,ht=2.5ex,dp=1.125ex,leftskip=.3cm,rightskip=.3cm plus1fil]{title in head/foot}%
    \usebeamerfont{title in head/foot}\insertshorttitle\hfill%
\end{beamercolorbox}%
\begin{beamercolorbox}[wd=.1\paperwidth,ht=2.5ex,dp=1.125ex,leftskip=.3cm,rightskip=.3cm plus1fil]{mycolor}%
\hfill\insertframenumber\,/\,\inserttotalframenumber
\end{beamercolorbox}}%
\vskip0pt%
}

%%%%%%%%%%%%%%%%%%%%%%%%%%%%%%%%%%%%%%%%%%%%%%%%%%%%%%%%%%%%%%%%%%%%%%
% Informações da apresentação
%%%%%%%%%%%%%%%%%%%%%%%%%%%%%%%%%%%%%%%%%%%%%%%%%%%%%%%%%%%%%%%%%%%%%%

\title[]{Um algoritmo baseado em programação dinâmica e renomeamento para minimização de formas normais}
\author{Matheus Pimenta}
\institute[UnB]{Universidade de Brasília}
\date{2016}

%%%%%%%%%%%%%%%%%%%%%%%%%%%%%%%%%%%%%%%%%%%%%%%%%%%%%%%%%%%%%%%%%%%%%%
% Diversos
%%%%%%%%%%%%%%%%%%%%%%%%%%%%%%%%%%%%%%%%%%%%%%%%%%%%%%%%%%%%%%%%%%%%%%

% Caminho para as imagens
\graphicspath{{./img/}}

% para mostrar as referencias com numeros
\setbeamertemplate{bibliography item}[text]

% Mostrar sumário entre seções
% tambem funciona com AtBeginSubsection
\AtBeginSection[]
{
  \begin{frame}<beamer>{Conteúdo}
    \tableofcontents[currentsection,currentsubsection]
  \end{frame}
}

%%%%%%%%%%%%%%%%%%%%%%%%%%%%%%%%%%%%%%%%%%%%%%%%%%%%%%%%%%%%%%%%%%%%%%
% Slides
%%%%%%%%%%%%%%%%%%%%%%%%%%%%%%%%%%%%%%%%%%%%%%%%%%%%%%%%%%%%%%%%%%%%%%

\begin{document}

\begin{frame}
\titlepage
\end{frame}

\begin{frame}{Conteúdo}
  \tableofcontents
\end{frame}

\indent

Lógicas têm sido utilizadas em Computação para representar e raciocinar sobre problemas. A representação se dá através de uma linguagem formal, que define a \emph{sintaxe} de uma lógica em particular, ou seja, a \emph{forma} dos enunciados que estão presentes na lógica. Cada palavra na linguagem formal é dita uma \emph{fórmula bem-formada}, ou, simplesmente, uma \emph{fórmula}.

Para cada lógica, é definida também uma \emph{semântica}, um instrumento para atribuir \emph{significado} às fórmulas. Isto é feito através da definição de diferentes \emph{interpretações}. Sob uma mesma interpretação, cada fórmula deve possuir um único significado. Em lógicas clássicas, como a \emph{lógica proposicional} e a \emph{lógica de primeira ordem}, o significado de uma fórmula sob uma interpretação deve ser somente um dentre dois valores possíveis: \emph{verdadeiro} ou \emph{falso}.

\emph{Satisfatibilidade}, o problema de determinar se existe uma interpretação sob a qual uma fórmula é verdadeira, é de grande interesse prático. Tal problema aparece, por exemplo, em vários desafios da microeletrônica, como síntese \cite{bloem2014sat}, otimização \cite{nieuwenhuis2006sat} e verificação de \textit{hardware} \cite{gupta2006sat}. Aparece também em problemas de raciocínio automático \cite{harrison2009handbook} e em muitos outros problemas relevantes \cite{horvitz1992automated}.

Satisfatibilidade é também de grande interesse teórico. Em 1971, Cook definiu a classe dos problemas \emph{NP-completos}, sendo satisfatibilidade proposicional o primeiro problema a ser descoberto como representativo desta classe. A partir destes resultados, Cook formalizou o enunciado do maior problema ainda não resolvido da Ciência da Computação:\break P \emph{versus} NP \cite{cook1971complexity}.

\emph{Validade}, o problema de determinar se uma dada fórmula é verdadeira sob \emph{qualquer} interpretação, é fortemente ligado ao problema da satisfatibilidade \cite{kleene68book}. Por esta forte relação, que detalhamos no Capítulo \ref{cap_referencial}, o problema da validade é também extensamente investigado.

Grande avanço já foi feito em direção a algoritmos de busca rápidos para satisfatibilidade e validade \cite{davis1960computing,davis1962machine,biere2009conflict}, apesar de ser conjecturado que qualquer algoritmo terá custo de tempo exponencial determinístico no pior caso \cite{cook1971complexity}.

Uma característica comum a diversos dos algoritmos para satisfatibilidade e validade é a redução do problema a fórmulas em uma determinada \emph{forma normal}. Uma forma normal é uma imposição de restrições sobre a forma de uma fórmula, ou seja, é um subconjunto das fórmulas de uma lógica. Formas normais criam vantagens ao lidar com problemas da lógica, pois fórmulas em formas mais restritas possuem mais propriedades em comum que podem ser exploradas, além do fato de que menos situações precisam ser consideradas pelos algoritmos.

Em algoritmos que utilizam formas normais, etapas de pré-processamento são necessárias, pois é preciso transformar uma fórmula qualquer para outra que esteja na respectiva forma normal. É importante ainda que o pré-processamento tenha custo de tempo polinomial determinístico. Somente assim a técnica implementada será de fato vantajosa, em vista da conjectura sobre o custo de tempo de qualquer algoritmo de busca para satisfatibilidade e validade.

Considerando a possibilidade de melhorar a eficiência total de pré-processamento e busca, conjectura-se que fórmulas menores produzem respostas mais rápido \cite{nonnengart2001computing}. O objetivo deste trabalho é testar esta hipótese através de experimentos. Em particular, investigamos algoritmos baseados na \emph{forma normal clausal}, definida formalmente no Capítulo \ref{cap_referencial}. Para obter fórmulas menores, implementamos algoritmos que reduzem o \emph{número de cláusulas} através de \emph{renomeamento}, conceitos também definidos no Capítulo \ref{cap_referencial}. Boy de la Tour \cite{de1992optimality} e Jackson et al. \cite{jackson2004clause} propõem algoritmos para este fim. No Capítulo \ref{cap_algoritmo}, propomos o nosso algoritmo para este fim, que permite ainda encontrar boas transformações restringindo o tamanho máximo do renomeamento. Comparamos o algoritmo proposto com o de Boy de la Tour no Capítulo \ref{cap_resultados} e, por fim, propomos trabalhos futuros no Capítulo \ref{cap_conclusao}.


\section{Referencial teórico}

\begin{frame}{Lógica proposicional}{Sintaxe}
	\vspace{-.2cm}
	\begin{footnotesize}
	\begin{block}{Símbolos proposicionais}
		$\mathcal{P} = \{a,b,...,a_1,a_2,...,b_1,b_2,... \}$ é dito o conjunto de \emph{símbolos proposicionais}.
	\end{block}
	\pause
	\vspace{-.2cm}
	\begin{block}{Fórmulas}
		Se $\phi \in \mathcal{P}$, então $\phi$ é uma \emph{fórmula}. Além disso, se $\phi_1,...,\phi_n$, $n \in \mathbb{N} \cup \{0 \}$, são fórmulas, então também são:
		\begin{enumerate}
			\pause\item \emph{Negação}: $\neg \phi_1$
			\pause\item \emph{Conjunção}: $\phi_1 \wedge ... \wedge \phi_n$
			\pause\item \emph{Disjunção}: $\phi_1 \vee ... \vee \phi_n$
			\pause\item \emph{Implicação}: $\phi_1 \rightarrow \phi_2$
			\pause\item \emph{Equivalência}: $\phi_1 \leftrightarrow \phi_2$
		\end{enumerate}
		\pause Denotamos o conjunto de fórmulas por $\mathcal{L}$.
	\end{block}
	\end{footnotesize}
\end{frame}

\begin{frame}{Lógica proposicional}{Sintaxe -- Exemplos}
	\begin{itemize}
		\item $\phi = (p \rightarrow q) \rightarrow \neg s$
		\pause\item $\psi = (p \vee q) \leftrightarrow (r \wedge s)$
		\pause\item $\xi  = \neg(p \rightarrow q)$
	\end{itemize}
\end{frame}

\begin{frame}{Lógica proposicional}{Sintaxe}
	\begin{block}{Subfórmulas imediatas}
		Na definição anterior, as fórmulas $\phi_i$ são \emph{subfórmulas imediatas}.
	\end{block}
	
	\pause
	\begin{block}{Subfórmulas}
		Dizemos que $\psi$ é subfórmula de $\phi$ se $\psi$ é subfórmula imediata de $\phi$, ou se $\psi$ é subfórmula de $\xi$ e $\xi$ é subfórmula imediata de $\phi$.
		
		\pause Notação: $\psi \sqsubset \phi$ e $\{\psi \mid \psi \sqsubset \phi \} = SF(\phi)$
	\end{block}
	
	\pause Exemplo: $\phi = (p \wedge q \wedge (r \rightarrow s))$\\
	\pause $p$, $q$ e $r \rightarrow s$ são subfórmulas imediatas de $\phi$.\\
	\pause $SF(\phi) = \{p,q,r \rightarrow s,r,s \}$
\end{frame}

\begin{frame}{Lógica proposicional}{Semântica}
	\vspace{-.3cm}
	\begin{footnotesize}
	\begin{block}{Valorações booleanas}
		Dizemos que $\mathbb{v}_0$ é uma \emph{valoração booleana} se $\mathbb{v}_0 : \mathcal{P} \longmapsto \{V,F \}$.
	\end{block}
	\pause
	\vspace{-.2cm}
	\begin{block}{Interpretações}
		Seja $\mathbb{v}_0$ é uma valoração booleana. Dizemos que $\mathbb{v} : \mathcal{L} \longmapsto \{V,F \}$ é uma \emph{interpretação definida por $\mathbb{v}_0$}, se:
		\begin{enumerate}
			\pause\item Se $\phi_1 \in \mathcal{P}$, então $\mathbb{v}(\phi_1) = \mathbb{v}_0(\phi_1)$.
			\pause\item $\mathbb{v}(\neg \phi_1) = V$ se, e somente se, $\mathbb{v}(\phi_1) = F$.
			\pause\item $\mathbb{v}(\phi_1 \wedge ... \wedge \phi_n) = V$ se, e somente se, $\mathbb{v}(\phi_i) = V$, para todo $i$.
			\pause\item $\mathbb{v}(\phi_1 \vee ... \vee \phi_n) = V$ se, e somente se, $\mathbb{v}(\phi_i) = V$, para algum $i$.
			\pause\item $\mathbb{v}(\phi_1 \rightarrow \phi_2) = V$ se, e somente se, $\mathbb{v}(\phi_1) = F$ ou $\mathbb{v}(\phi_2) = V$.
			\pause\item $\mathbb{v}(\phi_1 \leftrightarrow \phi_2) = V$ se, e somente se, $\mathbb{v}(\phi_1) = \mathbb{v}(\phi_2)$.
		\end{enumerate}
	\end{block}
	\end{footnotesize}
\end{frame}

\begin{frame}{Lógica proposicional}{Semântica -- Exemplos}
	Seja $\mathbb{v}$ definida por $\mathbb{v}_0 = \{(p,V),(q,V),(r,F) \}$ e considere $\phi = (\neg p \wedge q) \rightarrow r$. Então:
	\begin{enumerate}
		\pause\item $\mathbb{v}(\neg p) = F$
		\pause\item $\mathbb{v}(\neg p \wedge q) = F$
		\pause\item $\mathbb{v}(\phi) = \mathbb{v}((\neg p \wedge q) \rightarrow r) = V$
	\end{enumerate}
\end{frame}

\begin{frame}{Lógica proposicional}{Semântica -- Algumas definições}
	\begin{enumerate}
		\item Se existe $\mathbb{v}$ tal que $\mathbb{v}(\phi) = V$, dizemos que $\phi$ é \emph{satisfatível}.
		\pause\item Se existe $\mathbb{v}$ tal que $\mathbb{v}(\phi) = F$, dizemos que $\phi$ é \emph{falsificável}.
		\pause\item Se $\mathbb{v}(\phi) = V$ para toda $\mathbb{v}$, dizemos que $\phi$ é uma \emph{tautologia}.
		\pause\item Se $\mathbb{v}(\phi) = F$ para toda $\mathbb{v}$, dizemos que $\phi$ é uma \emph{contradição}, ou que $\phi$ é \emph{insatisfatível}.
		\pause\item Se $\phi$ é satisfatível e falsificável, dizemos que $\phi$ é uma \emph{contingência}.
	\end{enumerate}
\end{frame}

\begin{frame}{Lógica proposicional}{Semântica -- Exemplos}
	Seja $\phi$ uma fórmula qualquer, $\psi_1$ uma tautologia, $\psi_2$ uma contradição e $\psi_3$ uma contingência. Então:
	\begin{multicols}{2}
	\pause São tautologias:
	\begin{itemize}
		\pause\item $\phi \vee \neg \phi$
		\pause\item $\phi \rightarrow \phi$
		\pause\item $\phi \leftrightarrow \phi$
		\pause\item $\neg \psi_2$
	\end{itemize}
	\pause São contradições:
	\begin{itemize}
		\pause\item $\phi \wedge \neg \phi$
		\pause\item $\phi \leftrightarrow \neg \phi$
		\pause\item $\neg \psi_1$
	\end{itemize}
	\pause São contingências:
	\begin{itemize}
		\pause\item $p$
		\pause\item $\neg p$
		\pause\item $p \wedge q$
		\pause\item $p \vee q$
		\pause\item $p \rightarrow q$
		\pause\item $p \leftrightarrow q$
		\pause\item $\neg \psi_3$
	\end{itemize}
\end{multicols}
\end{frame}

\begin{frame}{Problemas da lógica proposicional}
	Seja $L \subseteq \mathcal{L}$. Se nos referimos a $L$ como um \emph{problema}, referimo-nos ao problema de, dada $\phi$ qualquer, determinar se $\phi \in L$ ou se $\phi \notin L$.
	\vspace{-.3cm}
	\begin{enumerate}
		\pause\item $\text{SAT} = \{\phi \in \mathcal{L} \mid \phi \text{ é satisfatível} \}$
		\pause\item $\text{UNSAT} = \{\phi \in \mathcal{L} \mid \phi \text{ é insatisfatível} \}\pause = \overline{\text{SAT}}$
		\pause\item $\text{VAL} = \{\phi \in \mathcal{L} \mid \phi \text{ é tautologia} \}$
	\end{enumerate}
	
	\vspace{.2cm}
	\pause Observações:
	\begin{enumerate}
		\pause\item Há algoritmos para SAT \cite{davis1960computing,davis1962machine,biere2009conflict}.
		\pause\item SAT e UNSAT são redutíveis um ao outro. \pause (claro!)
		\pause\item SAT e VAL são redutíveis um ao outro.
	\end{enumerate}
\end{frame}

\begin{frame}{Formas normais}{Regras de reescrita}
	Uma \emph{regra de reescrita} que transforma $\phi$ em $\psi$, escrito $\phi \longmapsto \psi$,
	\begin{enumerate}
		\pause\item \emph{preserva equivalência} se, e somente se, $\mathbb{v}(\phi) = \mathbb{v}(\psi), \forall \mathbb{v}$.
		\pause\item \emph{preserva satisfatibilidade} se, e somente se, $\phi,\psi \in \text{SAT}$ ou $\phi,\psi \notin \text{SAT}$.
	\end{enumerate}
\end{frame}

\begin{frame}{Formas normais}{Forma normal negada (FNN)}
	$$(p \wedge \neg q \wedge \neg r) \vee (x \wedge \neg y \wedge (r \vee s))$$
	
	\pause As transformações:
	\begin{enumerate}
		\pause\item $\neg \neg \phi_1 \longmapsto \phi_1$ \;\; (eliminação de dupla negação)
		\pause\item $\neg(\phi_1 \wedge ... \wedge \phi_n) \longmapsto \neg \phi_1 \vee ... \vee \neg \phi_n$ \;\; (De Morgan)
		\pause\item $\neg(\phi_1 \vee ... \vee \phi_n) \longmapsto \neg \phi_1 \wedge ... \wedge \neg \phi_n$ \;\; (De Morgan)
		\pause\item $\phi_1 \rightarrow \phi_2 \longmapsto \neg \phi_1 \vee \phi_2$
		\pause\item $\phi_1 \leftrightarrow \phi_2 \longmapsto (\phi_1 \rightarrow \phi_2) \wedge (\phi_2 \rightarrow \phi_1)$
	\end{enumerate}
	\pause preservam equivalência!
\end{frame}

\begin{frame}{Formas normais}{Forma normal clausal (FNC)}
	$$(p \vee \neg q \vee \neg r) \wedge (x \vee \neg y \vee r \vee s) \wedge (a \vee \neg b \vee c)$$
	
	\pause A transformação:
	\begin{center}
		$\phi \vee (\psi \wedge \xi) \longmapsto (\phi \vee \psi) \wedge (\phi \vee \xi)$ \;\; (distribuição)
	\end{center}
	\pause preserva equivalência!
	
	\pause Geralmente provoca crescimento exponencial.
\end{frame}

\begin{frame}{Renomeamento}
	\begin{enumerate}
		\item Escolhemos um conjunto de subfórmulas $R \subseteq SF(\phi)$.
		\pause\item Para cada $\psi \in R$:
		\begin{enumerate}
			\pause\item Escolhemos um símbolo proposicional novo $s(\psi) \in \mathcal{P}$.
			\pause\item Trocamos todas as ocorrências de $\psi$ por $s(\psi)$.
			\pause\item Incluímos a definição $s(\psi) \rightarrow \psi$ em conjunção.
		\end{enumerate}
	\end{enumerate}
	
	\begin{small}
	\pause Exemplo: $(\neg p_1 \wedge p_2 \wedge p_3) \vee (\neg q_1 \wedge q_2 \wedge \neg q_3) \vee (r_1 \wedge r_2 \wedge \neg r_3)$\\
	\pause Seja $\phi_1 = \neg p_1 \wedge p_2 \wedge p_3$ e $\phi_2 = \neg q_1 \wedge q_2 \wedge \neg q_3$.\\
	\pause Escolhendo $R = \{\phi_1,\phi_2 \}$, $s(\phi_1) = a$ e $s(\phi_2) = b$, temos\pause $$(a \vee b \vee (r_1 \wedge r_2 \wedge \neg r_3)) \wedge (a \rightarrow (\neg p_1 \wedge p_2 \wedge p_3)) \wedge (b \rightarrow (\neg q_1 \wedge q_2 \wedge \neg q_3))$$
	
	\pause Não preserva equivalência. \pause Mas preserva satisfatibilidade!
	\end{small}
\end{frame}

\begin{frame}{Reduzindo o número de cláusulas}{Contando cláusulas}
	Denotamos o \emph{número de cláusulas} geradas por $\phi$ ao ser colocada na FNC por $p(\phi)$.
	\pause
	\begin{center}
	\begin{tabular}{c|c}
		Forma de $\phi$                   & $p(\phi)$                                 \\ \hline
		$\phi_1 \wedge ... \wedge \phi_n$ & $p(\phi_1) + ... + p(\phi_n)$                                  \\
		$\phi_1 \vee ... \vee \phi_n$     & $p(\phi_1) \cdot ... \cdot p(\phi_n)$                       \\
		$x \text{ ou } \neg x, x \in \mathcal{P}$            & $1$                                                             \\
	\end{tabular}
	\end{center}
	
	\pause Exemplo: $\phi = (\neg p_1 \wedge p_2 \wedge p_3) \vee (\neg q_1 \wedge q_2 \wedge \neg q_3) \vee (r_1 \wedge r_2 \wedge \neg r_3)$\\
	\pause Temos que $$p(\phi) = (1 + 1 + 1)(1 + 1 + 1)(1 + 1 + 1) = 3^3 = 27$$
\end{frame}

\begin{frame}{Reduzindo o número de cláusulas}{O problema}
	\begin{block}{Problema}
		Escolher $R \subseteq SF(\phi)$ de modo que o número de cláusulas $p(\phi,R)$ da transformação por renomeamento seja mínimo.
	\end{block}
\end{frame}

\begin{frame}{Reduzindo o número de cláusulas}{Algoritmo de Boy de la Tour}
	\begin{block}{Árvores lineares}
		Seja $\phi$ uma fórmula na FNN. Se cada subfórmula de $\phi$ ocorre somente uma vez, dizemos que $\phi$ é uma \emph{árvore linear}.
	\end{block}
	
	\pause Se $\phi$ é uma árvore linear, o algoritmo de Boy de la Tour encontra um conjunto $R \subseteq SF(\phi)$ tal que $p(\phi,R)$ é ótimo (mínimo).
	
	\vspace{.1cm}
	\pause Seu custo de tempo no pior caso é $O(|SF(\phi)|^2)$.
\end{frame}

\begin{frame}{Reduzindo o número de cláusulas}{Algoritmo de Boy de la Tour}
	O algoritmo escreve o número de cláusulas na forma irredutível $$p(\phi) = c + a_\psi^\phi \cdot p(\psi)$$
	
	\pause Logo, se $R = \{\psi \}$, então $$p(\phi,R) = c + a_\psi^\phi + p(\psi)$$
	
	\pause Assim, é feita em $\phi$ uma busca em profundidade pré-ordem, incluindo cada $\psi \sqsubset \phi$ que satisfaz $$a_\psi^\phi \cdot p(\psi) > a_\psi^\phi + p(\psi)$$
\end{frame}

\begin{frame}{Reduzindo o número de cláusulas}{Algoritmo de Boy de la Tour -- Exemplo}
	$$\phi = (\neg p_1 \wedge p_2 \wedge p_3) \vee (\neg q_1 \wedge q_2 \wedge \neg q_3) \vee (r_1 \wedge r_2 \wedge \neg r_3)$$
	
	\pause
	\begin{figure}
		\centering
		
		\begin{tikzpicture}
		\tikzset{vertex/.style = {}}
		\tikzset{edge/.style = {->}}
		
		\node[vertex]  (1) at (   0,    1) {$\vee$};
		\node[vertex]  (2) at (  -3,    0) {$\wedge$};
		\node[vertex]  (3) at (   0,    0) {$\wedge$};
		\node[vertex]  (4) at (   3,    0) {$\wedge$};
		\node[vertex]  (5) at (  -4,   -1) {$\neg p_1$};
		\node[vertex]  (6) at (  -3,   -1) {$p_2$};
		\node[vertex]  (7) at (  -2,   -1) {$p_3$};
		\node[vertex]  (8) at (  -1,   -1) {$\neg q_1$};
		\node[vertex]  (9) at (   0,   -1) {$q_2$};
		\node[vertex] (10) at (   1,   -1) {$\neg q_3$};
		\node[vertex] (11) at (   2,   -1) {$r_1$};
		\node[vertex] (12) at (   3,   -1) {$r_2$};
		\node[vertex] (13) at (   4,   -1) {$\neg r_3$};
		
		\draw[edge] (1) to  (2);
		\draw[edge] (1) to  (3);
		\draw[edge] (1) to  (4);
		\draw[edge] (2) to  (5);
		\draw[edge] (2) to  (6);
		\draw[edge] (2) to  (7);
		\draw[edge] (3) to  (8);
		\draw[edge] (3) to  (9);
		\draw[edge] (3) to (10);
		\draw[edge] (4) to (11);
		\draw[edge] (4) to (12);
		\draw[edge] (4) to (13);
		\end{tikzpicture}
	\end{figure}
\end{frame}

\begin{frame}{Reduzindo o número de cláusulas}{Algoritmo de Boy de la Tour -- Exemplo}
	\begin{figure}
		\centering
		
		\begin{tikzpicture}
		\tikzset{vertex/.style = {}}
		\tikzset{edge/.style = {->}}
		
		\node[vertex]  (1) at (   0,    1) {$\vee$};
		\node[vertex]  (2) at (  -3,    0) {$\wedge$};
		\node[vertex]  (3) at (   0,    0) {$\wedge$};
		\node[vertex]  (4) at (   3,    0) {$\wedge$};
		\node[vertex]  (5) at (  -4,   -1) {$\neg p_1$};
		\node[vertex]  (6) at (  -3,   -1) {$p_2$};
		\node[vertex]  (7) at (  -2,   -1) {$p_3$};
		\node[vertex]  (8) at (  -1,   -1) {$\neg q_1$};
		\node[vertex]  (9) at (   0,   -1) {$q_2$};
		\node[vertex] (10) at (   1,   -1) {$\neg q_3$};
		\node[vertex] (11) at (   2,   -1) {$r_1$};
		\node[vertex] (12) at (   3,   -1) {$r_2$};
		\node[vertex] (13) at (   4,   -1) {$\neg r_3$};
		
		\node[vertex] (14) at (  -4,    1) {$\psi$};
		
		\node[vertex] (15) at (   4,    1) {$a_\psi^\phi = 1, p(\psi) = 27$};
		
		\draw[edge] (1) to  (2);
		\draw[edge] (1) to  (3);
		\draw[edge] (1) to  (4);
		\draw[edge] (2) to  (5);
		\draw[edge] (2) to  (6);
		\draw[edge] (2) to  (7);
		\draw[edge] (3) to  (8);
		\draw[edge] (3) to  (9);
		\draw[edge] (3) to (10);
		\draw[edge] (4) to (11);
		\draw[edge] (4) to (12);
		\draw[edge] (4) to (13);
		
		\tikzset{edge/.style = {->,red}}
		\draw[edge] (14) to (1);
		\end{tikzpicture}
	\end{figure}
\end{frame}

\begin{frame}{Reduzindo o número de cláusulas}{Algoritmo de Boy de la Tour -- Exemplo}
	\begin{figure}
		\centering
		
		\begin{tikzpicture}
		\tikzset{vertex/.style = {}}
		\tikzset{edge/.style = {->}}
		
		\node[vertex]  (1) at (   0,    1) {$\vee$};
		\node[vertex]  (2) at (  -3,    0) {$\wedge$};
		\node[vertex]  (3) at (   0,    0) {$\wedge$};
		\node[vertex]  (4) at (   3,    0) {$\wedge$};
		\node[vertex]  (5) at (  -4,   -1) {$\neg p_1$};
		\node[vertex]  (6) at (  -3,   -1) {$p_2$};
		\node[vertex]  (7) at (  -2,   -1) {$p_3$};
		\node[vertex]  (8) at (  -1,   -1) {$\neg q_1$};
		\node[vertex]  (9) at (   0,   -1) {$q_2$};
		\node[vertex] (10) at (   1,   -1) {$\neg q_3$};
		\node[vertex] (11) at (   2,   -1) {$r_1$};
		\node[vertex] (12) at (   3,   -1) {$r_2$};
		\node[vertex] (13) at (   4,   -1) {$\neg r_3$};
		
		\node[vertex] (14) at (  -4,    1) {$\psi$};
		
		\node[vertex] (15) at (   4,    1) {$a_\psi^\phi = 9, p(\psi) = 3$};
		
		\draw[edge] (1) to  (2);
		\draw[edge] (1) to  (3);
		\draw[edge] (1) to  (4);
		\draw[edge] (2) to  (5);
		\draw[edge] (2) to  (6);
		\draw[edge] (2) to  (7);
		\draw[edge] (3) to  (8);
		\draw[edge] (3) to  (9);
		\draw[edge] (3) to (10);
		\draw[edge] (4) to (11);
		\draw[edge] (4) to (12);
		\draw[edge] (4) to (13);
		
		\tikzset{edge/.style = {->,red}}
		\draw[edge] (14) to (2);
		\end{tikzpicture}
	\end{figure}
\end{frame}

\begin{frame}{Reduzindo o número de cláusulas}{Algoritmo de Boy de la Tour -- Exemplo}
	\begin{figure}
		\centering
		
		\begin{tikzpicture}
		\tikzset{vertex/.style = {}}
		\tikzset{edge/.style = {->}}
		
		\node[vertex]  (1) at (   0,    1) {$\vee$};
		\node[vertex]  (2) at (  -3,    0) {$a$};
		\node[vertex]  (3) at (   0,    0) {$\wedge$};
		\node[vertex]  (4) at (   3,    0) {$\wedge$};
		\node[vertex]  (8) at (  -1,   -1) {$\neg q_1$};
		\node[vertex]  (9) at (   0,   -1) {$q_2$};
		\node[vertex] (10) at (   1,   -1) {$\neg q_3$};
		\node[vertex] (11) at (   2,   -1) {$r_1$};
		\node[vertex] (12) at (   3,   -1) {$r_2$};
		\node[vertex] (13) at (   4,   -1) {$\neg r_3$};
		
		\node[vertex] (14) at (  -4,    1) {$\psi$};
		
		\node[vertex] (15) at (   4,    1) {$a_\psi^\phi = 9, p(\psi) = 1$};
		
		\draw[edge] (1) to  (2);
		\draw[edge] (1) to  (3);
		\draw[edge] (1) to  (4);
		\draw[edge] (3) to  (8);
		\draw[edge] (3) to  (9);
		\draw[edge] (3) to (10);
		\draw[edge] (4) to (11);
		\draw[edge] (4) to (12);
		\draw[edge] (4) to (13);
		
		\tikzset{edge/.style = {->,red}}
		\draw[edge] (14) to (2);
		\end{tikzpicture}
	\end{figure}
	
	$a \rightarrow (\neg p_1 \wedge p_2 \wedge p_3)$
\end{frame}

\begin{frame}{Reduzindo o número de cláusulas}{Algoritmo de Boy de la Tour -- Exemplo}
	\begin{figure}
		\centering
		
		\begin{tikzpicture}
		\tikzset{vertex/.style = {}}
		\tikzset{edge/.style = {->}}
		
		\node[vertex]  (1) at (   0,    1) {$\vee$};
		\node[vertex]  (2) at (  -3,    0) {$a$};
		\node[vertex]  (3) at (   0,    0) {$\wedge$};
		\node[vertex]  (4) at (   3,    0) {$\wedge$};
		\node[vertex]  (8) at (  -1,   -1) {$\neg q_1$};
		\node[vertex]  (9) at (   0,   -1) {$q_2$};
		\node[vertex] (10) at (   1,   -1) {$\neg q_3$};
		\node[vertex] (11) at (   2,   -1) {$r_1$};
		\node[vertex] (12) at (   3,   -1) {$r_2$};
		\node[vertex] (13) at (   4,   -1) {$\neg r_3$};
		
		\node[vertex] (14) at (  -4,    1) {$\psi$};
		
		\node[vertex] (15) at (   4,    1) {$a_\psi^\phi = 3, p(\psi) = 3$};
		
		\draw[edge] (1) to  (2);
		\draw[edge] (1) to  (3);
		\draw[edge] (1) to  (4);
		\draw[edge] (3) to  (8);
		\draw[edge] (3) to  (9);
		\draw[edge] (3) to (10);
		\draw[edge] (4) to (11);
		\draw[edge] (4) to (12);
		\draw[edge] (4) to (13);
		
		\tikzset{edge/.style = {->,red}}
		\draw[edge] (14) to (3);
		\end{tikzpicture}
	\end{figure}
	
	$a \rightarrow (\neg p_1 \wedge p_2 \wedge p_3)$
\end{frame}

\begin{frame}{Reduzindo o número de cláusulas}{Algoritmo de Boy de la Tour -- Exemplo}
	\begin{figure}
		\centering
		
		\begin{tikzpicture}
		\tikzset{vertex/.style = {}}
		\tikzset{edge/.style = {->}}
		
		\node[vertex]  (1) at (   0,    1) {$\vee$};
		\node[vertex]  (2) at (  -3,    0) {$a$};
		\node[vertex]  (3) at (   0,    0) {$b$};
		\node[vertex]  (4) at (   3,    0) {$\wedge$};
		\node[vertex] (11) at (   2,   -1) {$r_1$};
		\node[vertex] (12) at (   3,   -1) {$r_2$};
		\node[vertex] (13) at (   4,   -1) {$\neg r_3$};
		
		\node[vertex] (14) at (  -4,    1) {$\psi$};
		
		\node[vertex] (15) at (   4,    1) {$a_\psi^\phi = 3, p(\psi) = 1$};
		
		\draw[edge] (1) to  (2);
		\draw[edge] (1) to  (3);
		\draw[edge] (1) to  (4);
		\draw[edge] (4) to (11);
		\draw[edge] (4) to (12);
		\draw[edge] (4) to (13);
		
		\tikzset{edge/.style = {->,red}}
		\draw[edge] (14) to (3);
		\end{tikzpicture}
	\end{figure}
	
	$a \rightarrow (\neg p_1 \wedge p_2 \wedge p_3)$\\
	$b \rightarrow (\neg q_1 \wedge q_2 \wedge \neg q_3)$
\end{frame}

\begin{frame}{Reduzindo o número de cláusulas}{Algoritmo de Boy de la Tour -- Exemplo}
	\begin{figure}
		\centering
		
		\begin{tikzpicture}
		\tikzset{vertex/.style = {}}
		\tikzset{edge/.style = {->}}
		
		\node[vertex]  (1) at (   0,    1) {$\vee$};
		\node[vertex]  (2) at (  -3,    0) {$a$};
		\node[vertex]  (3) at (   0,    0) {$b$};
		\node[vertex]  (4) at (   3,    0) {$\wedge$};
		\node[vertex] (11) at (   2,   -1) {$r_1$};
		\node[vertex] (12) at (   3,   -1) {$r_2$};
		\node[vertex] (13) at (   4,   -1) {$\neg r_3$};
		
		\node[vertex] (14) at (  -4,    1) {$\psi$};
		
		\node[vertex] (15) at (   4,    1) {$a_\psi^\phi = 1, p(\psi) = 3$};
		
		\draw[edge] (1) to  (2);
		\draw[edge] (1) to  (3);
		\draw[edge] (1) to  (4);
		\draw[edge] (4) to (11);
		\draw[edge] (4) to (12);
		\draw[edge] (4) to (13);
		
		\tikzset{edge/.style = {->,red}}
		\draw[edge] (14) to (4);
		\end{tikzpicture}
	\end{figure}
	
	$a \rightarrow (\neg p_1 \wedge p_2 \wedge p_3)$\\
	$b \rightarrow (\neg q_1 \wedge q_2 \wedge \neg q_3)$
\end{frame}

\begin{frame}{Reduzindo o número de cláusulas}{Algoritmo de Boy de la Tour -- Exemplo}
	$$(a \vee b \vee (r_1 \wedge r_2 \wedge \neg r_3)) \wedge (a \rightarrow (\neg p_1 \wedge p_2 \wedge p_3)) \wedge (b \rightarrow (\neg q_1 \wedge q_2 \wedge \neg q_3))$$
	
	\pause Número de cláusulas: 9
\end{frame}


\section{O algoritmo}

\begin{frame}{Uma afirmação}
	\begin{block}{Afirmação}
		Seja $R \subseteq SF(\phi)$ um renomeamento ótimo entre os que contêm no máximo $j$ subfórmulas. Então $R - \{\psi \}$ é ótimo entre os que não consideram $\psi$ e contêm no máximo $j-1$ subfórmulas.
	\end{block}
	
	\pause Contraexemplo:\\
	\pause$\xi_1 = p_1 \wedge p_2 \wedge p_3 \wedge p_4$, \pause$\xi_2 = q_1 \wedge q_2$, \pause$\xi_3 = r_1 \wedge r_2$, \pause$\xi_4 = s_1 \wedge ... \wedge s_{100}$ \pause e $\phi = (\xi_1 \vee \xi_2) \wedge (\xi_3 \vee \xi_4)$\\
	\pause Então $p(\phi) = 208$ \pause e $R = \{\xi_1,\xi_4 \}$ é ótimo, com $p(\phi,R) = 108$.\\
	\pause Mas $R' = R - \{\xi_4 \}$ não é ótimo, pois \pause $$p(\phi,R') \pause = p(\phi,\{\xi_1 \}) \pause = 206 \pause > 110 \pause = p(\phi,\{\xi_3 \})$$
\end{frame}

\begin{frame}{Uma afirmação}
	\begin{center}
		Logo, a afirmação não é verdadeira.
		
		\pause Mas a usaremos como heurística!
	\end{center}
\end{frame}

\begin{frame}{Uma fórmula recursiva}
	Seja $SF(\phi) = \{\phi_1,...,\phi_n \}$ e denote por $f(i,j)$ um renomeamento ótimo entre os que contêm no máximo $j$ subfórmulas e consideram somente as subfórmulas em $\{\phi_1,...,\phi_i \}$. \pause Então $$f(i,0) = f(0,j) = \emptyset, \forall i,j$$ \pause e
	\begin{footnotesize}
	\[
	f(i,j) =
	\begin{cases} 
	\hfill f(i-1,j-1) \cup \{\phi_i \}   \hfill & \text{ se } p(\phi,f(i-1,j-1) \cup \{\phi_i \}) < p(\phi,f(i-1,j)) \\
	\hfill f(i-1,j) \hfill & \text{ caso contrário} \\
	\end{cases}
	\]
	\end{footnotesize}
	
	\pause Queremos $f(n,n)$!
\end{frame}

\begin{frame}{Uma implementação por computação ascendente}
	\begin{algorithmic}[1]
		\STATE seja $dp[0..n]$ um novo arranjo com $dp[j] = \emptyset$ para todo $j$
		\FOR{$i \gets 1$ \textbf{até} $n$}
			\FOR{$j \gets n$ \textbf{descendo até} $1$}
				\STATE $alt \gets dp[j-1] \cup \{\phi_i\}$
				\IF{$p(\phi,alt) < p(\phi,dp[j])$}
					\STATE $dp[j] \gets alt$
				\ENDIF
			\ENDFOR
		\ENDFOR
	\end{algorithmic}
	
	\pause O custo de tempo no pior caso é $O(|SF(\phi)|^3)$.
\end{frame}

\begin{frame}{Conjectura para árvores lineares}
	\begin{block}{Conjectura}
		Se $\phi$ é uma árvore linear e $SF(\phi) = \{\phi_1,...,\phi_n \}$, então $f(n,n)$ é ótimo.
	\end{block}
	
	\pause Apresentamos resultados experimentais para a conjectura na próxima seção.
\end{frame}


\section{Resultados experimentais}

\begin{frame}{Metodologia}{Representações de fórmulas}
	\begin{figure}
		\centering
		
		\raisebox{3.5\height}{$(p \leftrightarrow p) \leftrightarrow (p \leftrightarrow p)$}
		\hspace{.5cm}
		\begin{tikzpicture}
		\tikzset{vertex/.style = {}}
		\tikzset{edge/.style = {->}}
		
		\node[vertex]  (1) at (   0,    1) {$\leftrightarrow$};
		\node[vertex]  (2) at (  -1,    0) {$\leftrightarrow$};
		\node[vertex]  (3) at (   1,    0) {$\leftrightarrow$};
		\node[vertex]  (4) at (-1.5,   -1) {$p$};
		\node[vertex]  (5) at (-0.5,   -1) {$p$};
		\node[vertex]  (6) at ( 0.5,   -1) {$p$};
		\node[vertex]  (7) at ( 1.5,   -1) {$p$};
		
		\draw[edge] (1) to  (2);
		\draw[edge] (1) to  (3);
		\draw[edge] (2) to  (4);
		\draw[edge] (2) to  (5);
		\draw[edge] (3) to  (6);
		\draw[edge] (3) to  (7);
		\end{tikzpicture}
		\hspace{1cm}
		\begin{tikzpicture}
		\tikzset{vertex/.style = {}}
		\tikzset{edge/.style = {->}}
		
		\node[vertex]  (1) at (   0,    1) {$\leftrightarrow$};
		\node[vertex]  (2) at (   0,    0) {$\leftrightarrow$};
		\node[vertex]  (3) at (   0,   -1) {$p$};
		
		\draw[edge] (1) to [bend right] (2);
		\draw[edge] (1) to [bend left ] (2);
		\draw[edge] (2) to [bend right] (3);
		\draw[edge] (2) to [bend left ] (3);
		\end{tikzpicture}
		
		\hspace{1.2cm}Cadeia\hspace{2.3cm}Árvore sintática\hspace{1.8cm}DAG
	\end{figure}
\end{frame}

\begin{frame}{Metodologia}{Implementação}
	Foi implementado um programa em C++ 11 que realiza, em ordem, as seguintes transformações:
	\begin{enumerate}
		\pause\item Análise sintática
		\pause\item Conversão para FNN
		\pause\item Aplainamento
		\pause\item Conversão para DAG
		\pause\item Renomeamento
		\pause\item Conversão para FNC
	\end{enumerate}
\end{frame}

\begin{frame}{Metodologia}{Implementação -- Análise sintática}
	\begin{figure}
		\centering
		
		\raisebox{3.5\height}{$(p \leftrightarrow p) \leftrightarrow (p \leftrightarrow p)$}
		\hspace{1cm}
		\raisebox{7\height}{$\longmapsto$}
		\hspace{1cm}
		\begin{tikzpicture}
		\tikzset{vertex/.style = {}}
		\tikzset{edge/.style = {->}}
		
		\node[vertex]  (1) at (   0,    1) {$\leftrightarrow$};
		\node[vertex]  (2) at (  -1,    0) {$\leftrightarrow$};
		\node[vertex]  (3) at (   1,    0) {$\leftrightarrow$};
		\node[vertex]  (4) at (-1.5,   -1) {$p$};
		\node[vertex]  (5) at (-0.5,   -1) {$p$};
		\node[vertex]  (6) at ( 0.5,   -1) {$p$};
		\node[vertex]  (7) at ( 1.5,   -1) {$p$};
		
		\draw[edge] (1) to  (2);
		\draw[edge] (1) to  (3);
		\draw[edge] (2) to  (4);
		\draw[edge] (2) to  (5);
		\draw[edge] (3) to  (6);
		\draw[edge] (3) to  (7);
		\end{tikzpicture}
		
		\hspace{.5cm}Cadeia\hspace{4.6cm}Árvore sintática
	\end{figure}
\end{frame}

\begin{frame}{Metodologia}{Implementação -- Conversão para FNN}
	Coloca-se a fórmula na forma normal negada.
	\begin{enumerate}
		\pause\item Simplifica a implementação dos algoritmos de renomeamento e a conversão para FNC.
		\pause\item Permite testar nossa conjectura para árvores lineares.
	\end{enumerate}
\end{frame}

\begin{frame}{Metodologia}{Implementação -- Aplainamento}
	$$p \wedge (q \wedge r) \;\; \longmapsto \;\; p \wedge q \wedge r$$ e $$p \vee (q \vee r) \;\; \longmapsto \;\; p \vee q \vee r$$
	
	\pause Viabiliza mais simplificações!
\end{frame}

\begin{frame}{Metodologia}{Implementação -- Conversão para DAG}
	\begin{figure}
		\centering
		
		\begin{tikzpicture}
		\tikzset{vertex/.style = {}}
		\tikzset{edge/.style = {->}}
		
		\node[vertex]  (1) at (   0,    1) {$\leftrightarrow$};
		\node[vertex]  (2) at (  -1,    0) {$\leftrightarrow$};
		\node[vertex]  (3) at (   1,    0) {$\leftrightarrow$};
		\node[vertex]  (4) at (-1.5,   -1) {$p$};
		\node[vertex]  (5) at (-0.5,   -1) {$p$};
		\node[vertex]  (6) at ( 0.5,   -1) {$p$};
		\node[vertex]  (7) at ( 1.5,   -1) {$p$};
		
		\draw[edge] (1) to  (2);
		\draw[edge] (1) to  (3);
		\draw[edge] (2) to  (4);
		\draw[edge] (2) to  (5);
		\draw[edge] (3) to  (6);
		\draw[edge] (3) to  (7);
		\end{tikzpicture}
		\hspace{1cm}
		\raisebox{7\height}{$\longmapsto$}
		\hspace{1cm}
		\begin{tikzpicture}
		\tikzset{vertex/.style = {}}
		\tikzset{edge/.style = {->}}
		
		\node[vertex]  (1) at (   0,    1) {$\leftrightarrow$};
		\node[vertex]  (2) at (   0,    0) {$\leftrightarrow$};
		\node[vertex]  (3) at (   0,   -1) {$p$};
		
		\draw[edge] (1) to [bend right] (2);
		\draw[edge] (1) to [bend left ] (2);
		\draw[edge] (2) to [bend right] (3);
		\draw[edge] (2) to [bend left ] (3);
		\end{tikzpicture}
		
		\hspace{.5cm}Árvore sintática\hspace{3.5cm}DAG
	\end{figure}
\end{frame}

\begin{frame}{Metodologia}{Implementação -- Renomeamento}
	Executa-se um algoritmo para escolher $R$ (Boy de la Tour ou o que propomos) e aplica-se a transformação por renomeamento.
\end{frame}

\begin{frame}{Metodologia}{Implementação -- Renomeamento}
	No algoritmo de Boy de la Tour, quando $\psi = \psi_1 \vee ... \vee \psi_n$:
	
	\pause $$a_{\psi_i}^\phi = a_\psi^\phi \cdot \prod_{j \neq i} p(\psi_j)$$
	
	\pause Ao processar cada $\psi_i$, temos as seguintes alternativas:
	\begin{enumerate}
		\pause\item Calcular $a_{\psi_i}^\phi$ com um laço.
		\pause\item Calcular $a_\psi^\phi \cdot \prod_{j} p(\psi_j)$ antes e dividir por $p(\psi_i)$.
		\pause\item Calcular uma tabela de sufixos antes e combinar com um prefixo atualizado após cada iteração. \pause Fazemos assim!
	\end{enumerate}
\end{frame}

\begin{frame}{Metodologia}{Implementação -- Renomeamento}
	Além disso, pode ocorrer:
	
	\pause
	\begin{figure}
		\centering
		
		\begin{tikzpicture}
		\tikzset{vertex/.style = {}}
		\tikzset{edge/.style = {->}}
		
		\node[vertex]  (1) at (   0,    1) {$\psi$};
		\node[vertex]  (2) at (  -1,    0) {$\psi_1$};
		\node[vertex]  (3) at (   0,    0) {$\psi_2$};
		\node[vertex]  (4) at (   1,    0) {$\psi_3$};
		
		\draw[edge] (1) to  (2);
		\draw[edge] (1) to  (3);
		\draw[edge] (1) to  (4);
		\draw[edge] (2) to [bend right] (4);
		\end{tikzpicture}
	\end{figure}
	
	\pause Portanto, é necessária uma ordenação topológica:
	
	\pause
	\begin{figure}
		\centering
		
		\begin{tikzpicture}
		\tikzset{vertex/.style = {}}
		\tikzset{edge/.style = {->}}
		
		\node[vertex]  (1) at (   0,    1) {$\psi$};
		\node[vertex]  (2) at (  -1,    0) {$\psi_3$};
		\node[vertex]  (3) at (   0,    0) {$\psi_1$};
		\node[vertex]  (4) at (   1,    0) {$\psi_2$};
		
		\draw[edge] (1) to  (2);
		\draw[edge] (1) to  (3);
		\draw[edge] (1) to  (4);
		\draw[edge] (3) to  (2);
		\end{tikzpicture}
	\end{figure}
\end{frame}

\begin{frame}{Metodologia}{Implementação -- Conversão para FNC}
	Aplica-se distribuição para colocar a fórmula na FNC.
	
	\vspace{.5cm}
	\pause Opcionalmente, elimina-se literais e cláusulas repetidos e tautologias. \pause Exemplo:
	
	\pause $$(\neg p \vee q \vee \neg p) \wedge (r \vee \neg q) \wedge (\neg q \vee r) \wedge (p \vee \neg p)$$ $$\longmapsto$$ $$(\neg p \vee q) \wedge (r \vee \neg q)$$
\end{frame}

\begin{frame}{Metodologia}{Experimentos propostos}
	Sobre um \textit{benchmark} tradicional de 1200 fórmulas, foram executadas as seguintes combinações:
	
	\pause
	\vspace{.5cm}
	\begin{tabular}{l|cccccccccc}
		Combinação         & 1 & 2 & 3 & 4 & 5 & 6 & 7 & 8 & 9 & 10 \\ \hline
		Análise sintática  & X & X & X & X & X & X & X & X & X & X  \\
		Conversão para FNN & X & X & X & X & X & X & X & X & X & X  \\
		Aplainamento       & X & X & X & X & X & X & X & X & X & X  \\
		Conversão para DAG &   &   &   &   &   &   & X & X & X & X  \\
		Renomeamento       &   &   & 1 & 1 & 2 & 2 & 1 & 1 & 2 & 2  \\
		Conversão para FNC & 1 & 2 & 1 & 2 & 1 & 2 & 1 & 2 & 1 & 2  \\
	\end{tabular}
	
	\vspace{.4cm}
	\pause Em seguida, executamos um decisor de VAL baseado em FNC.
\end{frame}

\begin{frame}{Resultados e análise}{Combinações sem renomeamento}
	Combinação 1: sem renomeamento, sem simplificação\\
	Combinação 2: sem renomeamento, com simplificação
	
	\begin{itemize}
		\pause\item Na Combinação 1, 73\% excedeu limite de memória.
		\pause\item Na Combinação 2, 67\% excedeu limite de memória e 1\% excedeu limite de tempo.
		\pause\item Nos 27\% em que a transformação terminou em C1 e C2, 5 fórmulas (menos de 1\%) \emph{não} ficaram menores em C2.
	\end{itemize}
	
	\pause Simplificação é bom. \pause Mas não é suficiente!
\end{frame}

\begin{frame}{Resultados e análise}{Testando a conjectura para árvores lineares}
	Combinação 3: árvore, Boy de la Tour, sem simplificação\\
	Combinação 5: árvore, algoritmo proposto, sem simplificação
	
	\pause A transformação terminou em C3 e C5 para 49\% das fórmulas.
	
	\pause Nestes 49\%, Boy de la Tour e o algoritmo proposto produziram o mesmo número de cláusulas.
\end{frame}

\begin{frame}{Resultados e análise}{Comparações entre árvores e DAGs}
	Combinações $3,4,5,6$: árvore\\
	Combinações $7,8,9,10$: DAG
	
	\vspace{.1cm}
	\pause \begin{center}Comparando número de cláusulas.\end{center}
	\begin{scriptsize}
	\begin{tabular}{l|c|c|c|c|l}
		& $C_3 \times C_7$ & $C_4 \times C_8$ & $C_5 \times C_9$ & $C_6 \times C_{10}$ \\ \hline
		$C_i$ foi melhor em & 0 (0\%)     & 5 (0\%)     & 0 (0\%)     & 6 (1\%)      & fórmulas. \\
		$C_{i+4}$ foi melhor em   & 179 (15\%)  & 177 (15\%)  & 367 (31\%)  & 358 (30\%)   & fórmulas. \\
	\end{tabular}
	\end{scriptsize}
	
	\vspace{.1cm}
	\pause Claro, DAGs simplesmente permitem renomeamento global!
\end{frame}

\begin{frame}{Resultados e análise}{Comparações entre árvores e DAGs}
	Combinações $3,4,5,6$: árvore\\
	Combinações $7,8,9,10$: DAG
	
	\vspace{-.4cm}
	\pause
	\begin{center}
		\includegraphics[scale=.5]{arvore_x_dag_time}
	\end{center}
	
	\vspace{-.6cm}
	\pause Claro, DAG é uma estrutura mais compacta!
\end{frame}

\begin{frame}{Resultados e análise}{Comparações entre árvores e DAGs}
	Combinações $3,4,5,6$: árvore\\
	Combinações $7,8,9,10$: DAG
	
	\vspace{-.4cm}
	\pause
	\begin{center}
		\includegraphics[scale=.5]{arvore_x_dag_ptime}
	\end{center}
	
	\vspace{-.6cm}
	\pause Primeiro indício! \pause Converter para DAG é essencial.
\end{frame}

\begin{frame}{Resultados e análise}{Comparações entre os algoritmos de renomeamento}
	Combinações $7,8$: Boy de la Tour, sem e com simplificação\\
	Combinações $9,10$: Algoritmo proposto, sem e com simplificação
	
	\pause \begin{center}Comparando número de cláusulas.\end{center}
	\begin{itemize}
		\pause\item A transformação terminou em C7 e C9 para 73\% das fórmulas.
		\pause\item Em 3\%, a transformação terminou em C7 e C9 e C7 produziu menos cláusulas, com $\max \{|C7-C9| \} = 3$.
		\pause\item Em 8\%, a transformação terminou em C7 e C9 e C9 produziu menos cláusulas, com $\max \{|C7-C9| \} = 1.572.786$, onde C9 produziu 78 cláusulas.
	\end{itemize}
\end{frame}

\begin{frame}{Resultados e análise}{Comparações entre os algoritmos de renomeamento}
	Combinações $7,8$: Boy de la Tour, sem e com simplificação\\
	Combinações $9,10$: Algoritmo proposto, sem e com simplificação
	
	\vspace{-.4cm}
	\pause
	\begin{center}
		\includegraphics[scale=.5]{boy_x_knapsack_ptime}
	\end{center}
	
	\vspace{-.6cm}
	\pause Cada algoritmo leva vantagem em famílias de fórmulas específicas.
\end{frame}

\begin{frame}{Resultados e análise}{Tempo de busca em função do número de cláusulas}
	\begin{center}
		\includegraphics[scale=.45]{ptime}
	\end{center}
\end{frame}

\begin{frame}{Resultados e análise}{Tempo de busca em função do número de cláusulas}
	\begin{center}
		\includegraphics[scale=.6]{ptime_SYJ}
	\end{center}
	
	\pause Então, sim!
\end{frame}


\label{cap_conclusoes}

\indent

Neste trabalho, propomos a seguinte pergunta: fórmulas com menos cláusulas produzem respostas mais rápido? Como esforço para tentar responder esta pergunta, revisamos técnicas para reduzir o número de cláusulas geradas por uma fórmula proposicional, como as propostas por Boy de la Tour \cite{de1992optimality}, Nonnengart et al. \cite{nonnengart2001computing} e Jackson et al. \cite{jackson2004clause}, que são baseadas em renomeamento \cite{plaisted1986structure}. Após estudar estas técnicas, desenvolvemos também um algoritmo (Algoritmo \ref{knapsack}), baseado em programação dinâmica \cite{bellman2015applied}, para obter renomeamentos que geram poucas cláusulas. Propomos ainda experimentos para: verificar uma propriedade de optimalidade restrita deste algoritmo; comparar este algoritmo com o de Boy de la Tour; e tentar responder a pergunta levantada inicialmente. Dos resultados experimentais obtidos, apresentados no Capítulo \ref{cap_resultados}, tiramos as seguintes conclusões principais:
\begin{enumerate}
	\item O esforço de tentar reduzir o tamanho de uma fórmula ao convertê-la para uma forma normal compensa na grande maioria das vezes;
	\item A representação de fórmulas através de DAGs é altamente recomendável para algoritmos de minimização baseados em renomeamento;
	\item É bastante provável que o algoritmo que propomos produza renomeamentos ótimos para árvores lineares. Propomos, como trabalho futuro, provar esta afirmação analiticamente;
	\item Diferentes algoritmos de renomeamento levam vantagem em diferentes famílias de fórmulas. Em particular, para as famílias \emph{SYJ206} e \emph{SYJ212} \cite{raths07jar}, o algoritmo que propomos gera muito menos cláusulas que o de Boy de la Tour;
	\item Por fim, se consideramos fórmulas parecidas, ou seja, fórmulas com uma mesma estrutura principal em comum, as com menos cláusulas de fato produzem respostas mais rápido.
\end{enumerate}

Um ponto ainda não explorado do algoritmo proposto é sua capacidade de obter bons renomeamentos com um dado limite fixo para o número de subfórmulas que podem ser escolhidas. Observamos que esta propriedade pode ser bastante útil, dado que as complexidades dos algoritmos de busca para SAT e VAL crescem justamente com o número de símbolos proposicionais de uma fórmula \cite{davis1960computing,davis1962machine,biere2009conflict}. Ao limitar o número de subfórmulas que podem ser escolhidas para renomeamento, deixamos de reduzir completamente o número de cláusulas, mas também impedimos que o número de símbolos proposicionais da transformação resultante cresça demais. Como trabalho futuro, propomos encontrar uma maneira de determinar um ponto limite ótimo para o número de subfórmulas que podem ser escolhidas para renomeamento. Esta ideia está ilustrada no gráfico da Figura \ref{limitando}. Uma vantagem adicional que pode surgir ao explorar este ponto é o fato de que um dos fatores lineares da complexidade do algoritmo vem justamente do número máximo de subfórmulas que podem ser escolhidas. No limite em que este número tende a zero, a complexidade do algoritmo tende a ficar quadrática, como a complexidade do algoritmo de Boy de la Tour. Portanto, o trabalho futuro agora proposto tem potencial para reduzir os tempos de execução de ambas as etapas: pré-processamento e busca.

\begin{figure}
	\begin{center}
\begin{tikzpicture}
\begin{axis}[ymin=0,ymax=10,xmin=0,xlabel=Número de símbolos proposicionais,yticklabels={,,},xticklabels={,,},axis lines = left,legend entries={Complexidade de busca,Número mínimo de cláusulas,Ponto ótimo},legend style={at={(.95,1.3)}}]
\addplot[mark=none]{2^x};
\addplot[mark=none,dashed]{2^(4-x)};
\addplot[mark=*] coordinates {(2,4)};;
\end{axis}
\end{tikzpicture}
\end{center}
\label{limitando}
\caption{Ponto limite ótimo para o número de símbolos proposicionais.}
\end{figure}

Por último, observamos que o Algoritmo \ref{knapsack} se baseia principalmente na tomada de decisão da Linha 5. É provável que este critério possa ser apropriadamente substituído para resolver outros problemas, como minimização de outras formas normais, por exemplo. Como último trabalho futuro, deixamos a tarefa de investigar estas possíveis substituições.


\section{Referências}

\begin{frame}[allowframebreaks]
	\frametitle{Referências}
	\bibliographystyle{ieeetr}
	\bibliography{referencias}
\end{frame}

\section*{}

\begin{frame}
	\begin{center}
		Obrigado!\\
		matheuscscp@gmail.com
	\end{center}
\end{frame}

\begin{frame}{Problemas da lógica proposicional}
	Seja $A_{\text{UNSAT}}$ um algoritmo para UNSAT e $R_{\text{VAL}} =$\break ``Sobre a entrada $\phi \in \mathcal{L}$: Dê a resposta de $A_{\text{UNSAT}}$ sobre $\neg \phi$.''
	
	\begin{center}
		\begin{tabular}{c|c|c|c}
			$\phi$       & $\neg \phi$  & $A_{\text{UNSAT}}(\neg \phi)$ & $R_{\text{VAL}}(\phi)$ \\ \hline
			Tautologia   & Contradição  & Sim                            & Sim                          \\
			Contradição  & Tautologia   & Não                            & Não                          \\
			Contingência & Contingência & Não                            & Não                         
		\end{tabular}
	\end{center}
	
	\pause Seja $A_{\text{VAL}}$ um algoritmo para VAL e $R_{\text{UNSAT}} =$\break ``Sobre a entrada $\phi \in \mathcal{L}$: Dê a resposta de $A_{\text{VAL}}$ sobre $\neg \phi$.''
	
	\begin{center}
		\begin{tabular}{c|c|c|c}
			$\phi$       & $\neg \phi$  & $A_{\text{VAL}}(\neg \phi)$ & $R_{\text{UNSAT}}(\phi)$ \\ \hline
			Tautologia   & Contradição  & Não                            & Não                          \\
			Contradição  & Tautologia   & Sim                            & Sim                          \\
			Contingência & Contingência & Não                            & Não                         
		\end{tabular}
	\end{center}
\end{frame}

\end{document}
